\documentclass[12pt, a4paper]{article}
\usepackage{graphicx,color}
\usepackage[margin=0.5in]{geometry}
\usepackage{graphicx}
\usepackage[british]{babel}
\usepackage{mathtools}
\usepackage{amsfonts}
\usepackage{listings}
\graphicspath{ {img/} }
\DeclareMathSizes{20}{20}{20}{20}

\begin{document}

\title{\textbf{Joe Thompson Developing Internet-Based Applications First Assignment}}
\author{Joe Thompson}
\date{30th of October 2017}
\maketitle

\section{Summary of the Architecture}
\subsection{Enterprise Information System Tier}
The Enterprise Information System Tier consists of the default Rails DBMS, SQLite3. This is a relational database which stores the persistent data such as user details and post content.

\subsection{Views}
The View part of the MVC design pattern refers to what the user actually sees when using the application. This is represented by Rails by the \texttt{*html.erb} files in the \texttt{/views} directory which contains the templates which describe the HTML which will be generated by the application. In this application the views are primarily of the Post and User type which show the user the posts and user information as well as providing forms to upload new content and create new users.
\subsection{Controllers}
These are represented in the \texttt{/controllers} directory which contains the Ruby code to handle user interactions with the application. In MVC terms this is the component that converts user input into a form appropriate for the model where it can be stored and later displayed to the user in the form of a view. In this application, these are largely to do with accepting new posts and user information, as well as updating these.

\subsection{Model}
The model is concerned with the overall behaviour of the application. This is largely independent from the views presented to the user in order to promote maintainability and code reuse. This functionality is provided by Rails in the \texttt{/models} directory which is where the main classes are defined and linked to the database.

\end{document}
