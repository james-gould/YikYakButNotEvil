\documentclass[12pt, a4paper]{article}
\usepackage{graphicx,color}
\usepackage[margin=0.5in]{geometry}
\usepackage{graphicx}
\usepackage[british]{babel}
\usepackage{mathtools}
\usepackage{amsfonts}
\usepackage{listings}
\graphicspath{ {img/} }
\DeclareMathSizes{20}{20}{20}{20}

\begin{document}

\title{\textbf{MooseCast Client Specification \\ version 0.0.1}}
\author{MooseCast Team \\ \textit{Aaron Walker, Alex Toop and Joseph Thompson} \\ \\ \textcolor{red}{PRIVATE AND CONFIDENTIAL}}
\date{July 2017}
\maketitle

\tableofcontents
\clearpage

\section{Purpose of this Document}
The purpose of this document is to provide (in addition to the protocol specification) a complete set of requirements to which the MooseCast client application is to be developed against.

\section{Functional Requirements}
\subsection{Protocol Implementation}
The client

\section{Basic Client Specification}
\subsection{General UI}
The UI must display the posts on three separate feeds: the local feed, the home feed and the global feed. Each post will display the time passed since it was posted, the distance from the client in the local unit of distance and the number of total votes. Posts will be displayed sequentially (as in like Facebook Yik Yak or Instagram rather than Snapchat or Whisper). All toolbars and UI elements other than the posts themselves will move out of the way when scrolling down, showing a clean interface. A slider will allow the user to select the geographical range of posts they want to see.
\subsection{General Client Backend}
The backend must implement the MooseCast protocol outlined in this document and communicate with the TCP server to load posts. It will be implemented in Java on Android and Swift on iOS.
\subsection{General Theme}
The stylistic choices will be largely white or transparent with a single, changeable RGB colour for emphasis. A "night mode" will replace white with black for more comfortable viewing at night, and change based on either a toggle or automatically (default behaviour is toggle).
\subsection{Photo/Video Filters}
\subsubsection{Photographic Filters}
There will be a few generic filters to change white balance, colour tone ect., and also the option to make custom filters along the same lines as Instagram.
\subsubsection{Speedometer}
This filter will display a white analogue speedometer dial with the current GPS speed of the phone, it will have mph on the outer dial and km/h on the inner dial.
\subsubsection{Temperature}
This filter will display the current temperature with a white analogue dial, it will have degrees Celsius on the outer dial and degrees Fahrenheit on the inner dial.
\subsection{Language Support}
The initial version of MooseCast will be written in strict British English, avoiding Americanisms where possible. When ready for release, British English, American English, Spanish, French and German translations will be produced with more languages to follow. Welsh will be considered if the University offers support for MooseCast. If a translation is unavailable for a locale, the app should default to \textbf{British} English.
\clearpage 
\subsection{Units of Measurement}
The unit of measurement shown on the distance indicator is based on the locale of the device's OS, rather than location.
\subsubsection{Mile}
In the following locales, the distance indicator on each post shall be displayed using the international mile:

\begin{table}[ht]
\centering
\label{my-label}
\begin{tabular}{ll}
\textbf{Territory} & \textbf{Code}\\
United Kingdom & gb\\
United States & us\\
Liberia & lr\\
Burma & mm\\
American Samoa & as\\
Ascension Island & ac\\
Bahamas & bs\\
Belize & bz\\
British Virgin Islands & vg\\
Cayman Islands & ky\\
Dominica & dm\\
Falkland Islands & fk\\
Grenada & gd\\
Guam & gu\\
The N. Mariana Islands & mp\\
Samoa & ws\\
St. Lucia & lc\\
St. Vincent and The Grenadines & vc\\
St. Helena & sh\\
St. Kitts and Nevis & kn\\
Turks and Caicos Islands & tc\\
U.S. Virgin Islands & vi\\
\end{tabular}
\end{table}

\subsubsection{Scandinavian Mile}
In the following locales, the distance indicator will be displayed in Scandinavian miles (6.2 mi/10 km)
\begin{table}[ht]
\label{my-label}
\centering
\begin{tabular}{ll}
\textbf{Territory} & \textbf{Code}\\
Sweden & se\\
Norway & no\\
\end{tabular}
\end{table}

\subsubsection{Kilometre}
In all other locales, the kilometre shall be used for the distance indicator.
\clearpage










