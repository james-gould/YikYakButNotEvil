\documentclass[12pt, a4paper]{article}
\usepackage{graphicx,color}
\usepackage[margin=0.5in]{geometry}
\usepackage{graphicx}
\usepackage[british]{babel}
\usepackage{mathtools}
\usepackage{amsfonts}
\usepackage{listings}
\graphicspath{ {img/} }
\DeclareMathSizes{20}{20}{20}{20}

\begin{document}

\title{\textbf{Anonymoose}}
\author{Anonymoose Industries Ltd \\ \textit{Alex Toop, Lewis Sweeney and Joseph Thompson} \\ \\ \textcolor{red}{PRIVATE AND CONFIDENTIAL}}
\date{}
\maketitle

\tableofcontents
\clearpage

\section{Mission Statement}
Anonymoose is a mobile app which provides highly localised, fully anonymous and self-moderating posting and chat features. Posts from within a five mile radius are shown to the user, who may choose to initiate a chat with any other user who posts within this range. Anonymoose will fully support text, images, audio and video posts.

The goal of Anonymoose is to provide a digital space for local communities, particularly university towns to interact in a completely free and open manner, with no connection to their real life identity.
\clearpage
\section{UI and Feature Description}
\subsection{Units of Measurement}
The unit of measurement shown on the distance indicator is based on the locale of the device's OS.
\subsubsection{Mile}
In the following locales, the distance indicator on each post shall be displayed using the international mile:

\begin{table}[h]
\centering
\label{my-label}
\begin{tabular}{ll}
\textbf{Territory} & \textbf{Code}\\
United Kingdom & gb\\
United States & us\\
Liberia & lr\\
Burma & mm\\
American Samoa & as\\
Ascension Island & ac\\
Bahamas & bs\\
Belize & bz\\
British Virgin Islands & vg\\
Cayman Islands & ky\\
Dominica & dm\\
Falkland Islands & fk\\
Grenada & gd\\
Guam & gu\\
The N. Mariana Islands & mp\\
Samoa & ws\\
St. Lucia & lc\\
St. Vincent and The Grenadines & vc\\
St. Helena & sh\\
St. Kitts and Nevis & kn\\
Turks and Caicos Islands & tc\\
U.S. Virgin Islands & vi\\
\end{tabular}
\end{table}

\subsubsection{Scandinavian Mile}
In the following locales, the distance indicator will be displayed in Scandinavian miles (6.2 mi/10 km)
\begin{table}[h]
\centering
\label{my-label}
\begin{tabular}{ll}
\textbf{Territory} & \textbf{Code}\\
Sweden & se\\
Norway & no\\
\end{tabular}
\end{table}

\subsubsection{Kilometre}
In all other locales, the kilometre shall be used for the distance indicator. 


\clearpage
\section{Technical Description}
\subsection{Anatomy of an Anonymoose Post}
Post data is sent between the client app and the server using the following format:
\begin{table}[h]
\centering
\label{my-label}
\begin{tabular}{lll}
\textbf{Byte Offset} & \textbf{Data Type} &\textbf{Description}\\
0 & u8 array & The UTF-8 string ``POST''\\
4 & i64 & a unique post identifier\\
12 & i32 & the UNIX timestamp of the post date\\
16 & f32 & the latitude of the post location in decimal degrees\\
20 & f32 & the longitude of the post location in decimal degrees\\
24 & i16 & the number of upvotes\\
26 & i16 & the number of downvotes\\
28 & bool[12] & bytes for various flags\\
40 & i64 array & null-terminated array of post IDs of replies\\
variable & u8 array & null-terminated UTF-8 string up to 256 chars\\
variable & raw bytes & any additional data (eg photos, videos, sound) terminated by EOF\\

\end{tabular}
\end{table}

\subsection{Status Codes / AM protocol}
\begin{table}[h]
\centering
\label{my-label}
\begin{tabular}{ll}
\textbf{Code} &\textbf{Description}\\
100 & initial connection\\
101 & client submitting post\\
102 & client submitting vote\\
103 & client requesting deletion\\
104 & client reporting a post\\
&\\
200 & server standing by for IO\\
201 & operation successful\\
202 & closing connection\\
&\\
300 & unspecified error\\
301 & connection timeout\\
302 & post format is corrupt\\
303 & database error\\
304 & voting error\\
\end{tabular}
\end{table}

\subsection{Operation Description}
\begin{table}[h]
\centering
\label{my-label}
\begin{tabular}{ll}
\textbf{Operation} &\textbf{Description}\\
upvote & after code 102, a zero byte followed by the post ID\\
downvote & after code 102, a non-zero byte followed by the post ID\\
delete & after code 103, the post ID\\
report & after code 104, the post ID\\
\end{tabular}
\end{table}

\clearpage
\subsection{Algorithms for Calculating Distance}
\subsubsection{Imposing the Five Mile Limit}
`Under the bonnet', Anonymoose uses decimal degrees of latitude and longitude to calculate the position. The client reports its current position and the server returns the 30 latest posts within five miles. The algorithm for performing this is thus:
\begin{verbatim}
/*get the latitude and longitude from the location given by the client*/

float current_latitude = get_GPS_latitude();
float current_longitude = get_GPS_longitude();

/*degrees of latitude are approx. 69 mi and we want a limit of 5 mi*/ 

float min_latitude = current_latitude - 0.073;
float max_latitude = current_latitude + 0.073;

/*degrees of longitude are more complex, in order to impose our 5 mi limit we need
to use some trigonometry and the current latitude. First we get the length of a
longitudinal degree in miles*/

float miles_in_longitudinal_degree = cos(current_latitude) * 69;

/*now we need to figure out how many longitudinal degrees are in five miles
at the current latitude*/

float longitudinal_degrees_per_5_miles =  (1 / miles_in_longitudinal_degree) * 5;

min_longitude = current_longitude - longitudinal_degrees_per_5_miles;
max_longitude = current_longitude + longitudinal_degrees_per_5_miles;

/*assuming the database is correctly sorted in order of post times, we search it for
posts in the corresponding range*/

int counter = 0;
struct posts_array[30];
foreach(posts_in_database) {
  if(current_post.latitude < max_latitude && currentpost.latitude > min_latitude
  && post.longitude < max_longitude && post.longitude > min_longitude) {
    postsArray[counter] = current_post;
    counter++;
  }
  if(counter == 30) {
    break;
  }
}

return posts_array;

\end{verbatim}


\end{document}











